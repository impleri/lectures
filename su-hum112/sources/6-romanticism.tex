%%%%%%%%%%%%%%%%%%%%%%%%%%%%%%%%%%%%
%% 
%% Romanticism and Realism: 1750--1850
%% 
%% Key Events:
%% Industrial Revolution, French Revolution, Civil War
%% 
%% Key Ideas:
%% Transcendentalism, Slavery, Socialism, Evolution, Nationalism
%% 
%% Style: Neoclassicism, Romanticism, Realism
%% 
%% Figures:
%% Goya, Manet, Beethoven, Wagner
%% Goethe, Dickens, Brönte, Tolstoy, Melville, Douglass, Twain
%% Kant, Hegel, Schopenhauer, Marx, Darwin
%% 
%%%%%%%%%%%%%%%%%%%%%%%%%%%%%%%%%%%%

\firstslde[8 February 2012]{Romanticism and Realism: 1750--1850}
% Quiz Results Slide

\section{Results}
\begin{frame}{Quiz \#1}
	\begin{tabular}{r || c}
		Grade	& Count	\\
		A		& 4		\\
		B		& 2		\\
		C		& 1		\\
		D		& 0 	\\
		F		& 5		\\
	\end{tabular}
\end{frame}



\section{Creating States}


\subsection{Reign of Wars}
\begin{frame}{American Wars}
	\begin{itemize}
		\item<10->1812--1815. War of 1812 (Second American Revolution)
	\end{itemize}
\end{frame}

\subsection{A New World}
\begin{frame}{New Practices}
	\begin{itemize}
		\item<1->The rise of chattel slavery (Native Americans, Africans)
		\item<2->``Taxation without Representation''
		\item<3->Political integration and coercion
		\item<4->Republicanism
		\item<5->Confederation
		\item<6->Civil religion
		\item<7->Inalienable individual rights
		\item<8->Exceptionalism
		\item<9->Monroe Doctrine (1823)
		\item<10->Manifest Destiny
	\end{itemize}
\end{frame}

\section{European Shifts}

\subsection{Revivals}
\begin{frame}{Romanticism}
 	\slide{2}{\emph{The Third of May 1808} [1814] by Goya}{img-third_may.jpeg}
 	\slide{14}{\emph{Luncheon on the Grass} [1862--1863] by Manet}{img-luncheon.jpeg}
	\begin{itemize}
		\item<1,3-13>Francisco Goya [1746--1828]
		\item<3-13>Johann Wolfgang von Goethe [1749--1832] (\emph{Faust} [1806--1832])
		\item<4-13>Ludwig van Beethoven [1770--1827] (\emph{Symphony No. 9} [1824])
		\item<5-13>Samuel Taylor Coleridge [1772--1834] (\emph{Kubla Khan} [1797])
		\item<6-13>Arthur Schopenhauer [1788--1860] (\emph{World as Will and Representation} [1818])
		\item<7-13>John Keats [1795--1821] (\emph{To Autumn} [1819])
		\item<8-13>Ralph Waldo Emerson [1803--1882] (``Self-Reliance'' [1841])
		\item<9-13>Margaret Fuller [1810--1850] (\emph{Woman in the Nineteenth Century} [1843])
		\item<10-13>Richard Wagner [1813--1883] (\emph{Die Walk{\"u}rie} [1870])
		\item<11-13>Henry David Thoreau [1817--1862] (\emph{Walden} [1854], ``Civil Disobedience'' [1849])
		\item<12-13>Walt Whitman [1819--1892] (\emph{O Captain, My Captain} [1865])
		\item<13>{\'E}douard Manet [1832--1883]
	\end{itemize}
\end{frame}

\subsection{New Nations}
\begin{frame}{A Changed Europe}
	\slide{5}{After the Congress of Vienna}{map-1815-europe.jpeg}
	\begin{itemize}
		\item<1-4,6->1803. United Kingdom of Great Britain and Ireland
		\item<2-4,6->1804--1814. Napoleonic France dominates Europe
		\item<3-4,6->1806. End of Holy Roman Empire
		\item<4,6->1815. Battle of Waterloo. Constitutional monarchy (Louis XVII) established in France
		\item<6->1830--1848. July Revolution and Monarchy in France (Louis-Philippe I)
		\item<7->1830--1831. Belgian Revolution from Netherlands
		\item<8->1848. Revolution(s) of 1848: French Second Republic and French Second Empire with Napoleon III (1851--1871)
		\item<9->1848, 1860, 1870. Italian Unification
		\item<10->1870--1871. Franco-Prussian War. German Unification (First Reich) and Third French Republic.
		\item<11->Second Industrial Revolution
	\end{itemize}
\end{frame}
