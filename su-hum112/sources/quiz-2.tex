\documentclass[12pt]{examdesign}
\NoRearrange
\OneKey
\NumberOfVersions{3}
\ConstantBlanks{0.5in}
\class{HUM 112: World Cultures 2}
\examname{Quiz 2}

\begin{document}
\begin{exampreface}
	\textbf{Unless noted otherwise, all questions are worth 2 points each.}
\end{exampreface}

% 5 questions = 10 points
\begin{truefalse}[title={Statements}]
	In this section, you must determine if the statement is \emph{completely true} or false. Be sure to spell out the word ``true'' or ``false'' for clarity. 

	\begin{question}
		\answer{True} Captain James Cook was the first Englishman to discover the Easter Islands.
	\end{question}

	\begin{question}
		\answer{False} Great Britain's first attempt to settle in present-day Virginia was a smashing success (hint: Roanake).
	\end{question}

	\begin{question}
		\answer{False} Ralph Waldo Emerson's essay ``On Self-Reliance'' became a central part of German nationalist culture.
	\end{question}

	\begin{question}
		\answer{False} John Locke's \emph{On Human Understanding} argues that a monarchy is necessary for the social order.
	\end{question}

	\begin{question}
		\answer{True} Diego Vel\'{a}zquez was able to paint \emph{Venus at her Mirror} (``Robekey Venus'') in conservative Spain because he was one of the painters for the royal court.
	\end{question}
\end{truefalse}

% 5 questions (x4) = 20 points
\begin{shortanswer}[title={Short Answers}]
	The following questions can be answered with just one or two words. Each are worth 4 points.

	\begin{question}
		Which famous German composer of the 18th and/or 19th century was deaf by the end of his life yet still composed music? Name one of his famous pieces (at least two were discussed in class).
		\examvspace{0.5 in}
		\begin{answer}
			(1) Beethoven. (2) \emph{Symphony No. 5, 9}
		\end{answer}
	\end{question}

	\begin{question}
		Who are two major American poets connected with the 19th century transcendentalism movement?
		\examvspace{0.5 in}
		\begin{answer}
			Emerson, Thoreau, Whitman, Fuller
		\end{answer}
	\end{question}

	\begin{question}
		Which French king revitalized ballet? Name one thing, place, person, or name associated with him.
		\examvspace{0.5 in}
		\begin{answer}
			(1) Louis XIV. (2) ``Sun King'', Versailles, Rubens, Poussin, Lully
		\end{answer}
	\end{question}

	\begin{question}
		What was the name the Third Estate used to identify itself after the Tennis Court Oath? (Hint: This group drafted and passed the Declaration of the Rights of Man and Citizen...)
		\examvspace{0.5 in}
		\begin{answer}
			National Assembly
		\end{answer}
	\end{question}

	\begin{question}
		Which city had city had established itself by 1800 as the center of musical development? (Hint: It is the capital of a European country where the hills are alive...)
		\examvspace{0.5 in}
		\begin{answer}
			Vienna
		\end{answer}
	\end{question}
\end{shortanswer}

% 5 questions = 10 points
\begin{multiplechoice}[title={Multiple Choice}]
	\begin{question}
		Which musical term is used to indicate a solo musical piece without vocal accompaniment, either as a single instrument or a group? (Hint: it comes from Latin word meaning ``to sound'')
		\choice{Cantanta}
		\choice[!]{Sonata}
		\choice{Fugue}
		\choice{Symphony}
	\end{question}

	\begin{question}
		Which of the following \emph{did not} occur as part of the French Revolution?
		\choice{De-Christianization}
		\choice{Slavery banned in the colonies}
		\choice[!]{Sale of territory named Louisiana to the English}
		\choice{Beheading of Marie Antoinette}
	\end{question}

	\begin{question}
		Which nineteenth century German philosopher wrote about poor labor conditions in factories?
		\choice[!]{Karl Marx}
		\choice{Immanuel Kant}
		\choice{G. W. F. Hegel}
		\choice{David Hume}
	\end{question}

	\begin{question}
		While Jonathan Swift's essay ``A Modest Proposal'' proposes wealthy English families should actually eat Irish children, what topic is he addressing?
		\choice{Over-population}
		\choice[!]{English oppression}
		\choice{Poverty in Ireland}
		\choice{Cannibalism}
	\end{question}

	\begin{question}
		What ongoing process or event during the eighteenth and nineteenth centuries make possible the concept of a factory?
		\choice[!]{Industrial Revolution}
		\choice{Printing press}
		\choice{Over-population}
		\choice{Slavery}
	\end{question}
\end{multiplechoice}

% 5 questions = 10 points
\begin{matching}[title={Matching}]
	Match the following musicians with their works. Each answer will be used only once.

	\pair{J. S. Bach}{\emph{Brandenburg Concerto}, \emph{Toccata and Fugue in Dm}}
	\pair{Joseph Haydn}{\emph{Symphony No. 94} (\emph{``Surprise'' Symphony})}
	\pair{Wolfgang Amadeus Mozart}{\emph{Symphony no. 40}, \emph{Marriage of Figaro}}
	\pair{Ludwig van Beethoven}{\emph{Symphony No. 5}, \emph{Symphony No. 9} (\emph{Ode to Joy})}
	\pair{Richard Wagner}{\emph{Der Ring des Nibelungen} (\emph{The Ring of Nibelungen}), \emph{Die Walk{\"u}rie} (\emph{The Valkyrie})}
\end{matching}

% 10 questions = 20 points
\begin{fillin}[title={People}]
	In this section, match the person to their field of contribution \emph{during the Enlightenment and Exploration era (1650--1850)}. Your choices are as follows: \textbf{A} for Art; \textbf{L} for Literature; \textbf{P} for Philosophy, Religion, and Politics; \textbf{N} for None.

	\begin{question}
		\blank{A}  Nicolas Poussin
	\end{question}

	\begin{question}
		\blank{P}  Thomas Hobbes
	\end{question}

	\begin{question}
		\blank{N}  Martin Luther
	\end{question}

	\begin{question}
		\blank{L}  John Milton
	\end{question}

	\begin{question}
		\blank{P}  Jean-Jacques Rousseau
	\end{question}

	\begin{question}
		\blank{N}  David Beckham
	\end{question}

	\begin{question}
		\blank{P}  Maximilien Robespierre
	\end{question}

	\begin{question}
		\blank{L}  Margaret Fuller
	\end{question}

	\begin{question}
		\blank{A}  Francisco Goya
	\end{question}

	\begin{question}
		\blank{L}  John Keats
	\end{question}
\end{fillin}

\pagebreak
% 1 question (x10) = 10 points
\begin{shortanswer}[title={Short Essay}]
	The following question must be answered in one to two paragraphs. Use the back of this page if space is needed. It is worth 10 points. I will be grading on the following criteria: (1) correctly identify and very briefly describe the thing in question and (2) plausibly provide an argument for its relevance today.

	\begin{question}
		Briefly explain one revolution which occurred during the eighteenth or nineteenth century (we discussed at least seven) and, in your own opinion, explain how it has effected our life today.
		\examvspace{5 in}
		\begin{answer}
			Glorious revolution, American revolution, French revolution, ``Copernican revolution'', Industrial revolution, Women's rights, Abolition.
		\end{answer}
	\end{question}
\end{shortanswer}

\pagebreak
% 5 questions = 0 points
\begin{shortanswer}[title={Bonus Section}]
	The following questions are optional. You will receive two additional points for each one answered correctly.

	\begin{question}
		Who wrote \emph{Gulliver's Travels}?
		\examvspace{0.5 in}
		\begin{answer}
			Jonathan Swift
		\end{answer}
	\end{question}

	\begin{question}
		Which type of musical work is performed by a choir without movement or scenery and is accompanied by music?
		\examvspace{0.5 in}
		\begin{answer}
			Oratorio
		\end{answer}
	\end{question}

	\begin{question}
		Which term indicates a form of slavery in which slaves are considered owned property?
		\examvspace{0.5 in}
		\begin{answer}
			Chattel
		\end{answer}
	\end{question}

	\begin{question}
		Which war broke out in the Americas which was part of the global fighting related to the Seven Years War in Europe?
		\examvspace{0.5 in}
		\begin{answer}
			French and Indian War
		\end{answer}
	\end{question}

	\begin{question}
		Which person was the central focus in Walt Whitman's \emph{O Captain, My Captain}?
		\examvspace{0.5 in}
		\begin{answer}
			Abraham Lincoln
		\end{answer}
	\end{question}
\end{shortanswer}

\end{document}
