%%%%%%%%%%%%%%%%%%%%%%%%%%%%%%%%%%%%
%% 
%% Dealing with Global Wars: 1900--2000
%% 
%% Key Events:
%% Bolshevik Revolution, World Wars, Great Depression, Holocaust, Harlem Renaissance
%% 
%% Key Ideas:
%% 
%% 
%% Style: Surrealism, Expressionism, Pop
%% 
%% Figures:
%% Picasso, Pollock, Warhol
%% Cullen, Hamilton
%% Jung, Sartre
%% 
%% 
%%%%%%%%%%%%%%%%%%%%%%%%%%%%%%%%%%%%

\firstslide[29 February 2012]{Dealing with Global Wars: 1900--2000}

\section{The War to End All Wars}
\subsection{Background}
\begin{frame}{Political Background}
	\slide{2}{European Alliances in 1914}{map-1914-europe.pdf}
	
	\begin{itemize}
		\item<1,3->Europe maintained peace during late 19th through a number of interlocking peace treaties (``Triple Entente'')
		\item<3->Ottoman Empire begins to fall apart and recede from eastern Europe and the Balkans
		\item<4->Multiple wars within the Balkans (both against the Ottomans and each other) made a very unstable region
		\item<5->Growing tension between the Central Powers (Germany, Italy, and Austro-Hungarian Empire) and Russia in the Balkans
		\item<6->28 June 1914. The heir to the Austro-Hungarian throne is assassinated in Sarajevo.
		\item<7->Diplomatic crises arose, further straining tensions between Austro-Hungarian Empire and Russia (through its ally, Serbia)
		\item<8->Russia began mobilizing troops to Serbia on 29 July. Germany mobilized its troop on 30 July and declared war on Russia one day later.
		\item<9->France mobilized its troops against Germany on 1 August because of its losses during the Franco-Prussian wars in 1860s.
		\item<10->UK declared war against Germany on 4 August.
	\end{itemize}
\end{frame}

\subsection{The Great War}
\begin{frame}{The Great War}
	\slide{7}{Europe after the Great War}{map-1923-europe.pdf}
	
	\begin{itemize}
		\item<1-6>1915. German U-boat sank the British liner \emph{Lusitania} (which had many American passengers). Pres. Wilson demanded that passenger ships are excluded from military targets, and Germany complied.
		\item<2-6>1917. Zimmerman Telegram. Germany asked Mexico to join in the fight in return for support to regain territories lost to the US during the Mexican-American War (1840s).
		\item<3-6>November 1917. Bolshevik Revolution.
		\item<4-6>3 March 1918. Russia and Germany agree to a treaty and cease attacking. Allies send a small invasion to Russia in order to support the monarchy.
		\item<5-6>4 November 1918. German Revolution began.
		\item<6>11 November 1918. Armistice signed. Cease-fire enacted at 11:11am.
	\end{itemize}
\end{frame}

\subsection{Cultural Impact}
\begin{frame}{General Feelings}
	\begin{itemize}
		\item<+->Heroism loses its glamor.
		\item<+->Massive amounts of people faced the ``horrors of war'' for the first time.
		\item<+->Armistice Day. Remembering the atrocities experienced.
		\item<+->Decoupling of artistic works from ``real world.''
	\end{itemize}
\end{frame}

\begin{frame}{Literary Developments}
	\slide{11}{\emph{The Persistence of Memory} [1931] by Salvador Dal{\'i}}{img-persistence_memory.jpeg}
	\slide{12}{\emph{Swans Reflecting Elephants} [1937] by Salvador Dal{\'i}}{img-swans-elephants.jpeg}
	\begin{itemize}
		\item<1-10>Marcel Proust [1871--1922] (\emph{In Search of Lost Time} [1913--1927], \emph{Swann's Way} [1913])
		\item<2-10>James Joyce [1882--1941] (\emph{Ulysses} [1922], \emph{Finnegan's Wake} [1939])
		\item<3-10>Virginia Woolf [1882--1941] (\emph{Mrs. Dalloway} [1925])
		\item<4-10>Franz Kafka [1883--1924] (\emph{The Metamorphosis} [1915])
		\item<5-10>T.S. Eliot [1888-1965] (\emph{The Waste Land} [1921]).
		\item<6-10>Wilfred Own [1893--1918] (``Dulce et Decorum Est'' [1918]).
		\item<7-10>Ernest Hemmingway [1899--1961] (\emph{A Farewell to Arms} [1929], \emph{For Whom the Bell Tolls} [1940])
		\item<8-10>Dada (anti-?) and Surrealism
		\item<9-10>Marcel Duchamp [1887--1968]
		\item<10>Salvador Dal{\'i} [1904--1989]
	\end{itemize}
\end{frame}

\section{America Between Wars}
\begin{frame}{Dealing with The Great Depression}
	\begin{itemize}
		\item<+->1919. Riots across America because of racial violence (``Red Summer'' in Chicago, Washington DC, and Elaine, AR)
		\item<+->1920. Prohibition passed
		\item<+->October 1927. Stock Market Crash. (Black Thursday, the 24th).
		\item<+->1933. Beginning of the New Deal (Glass-Steagall, FDIC, FHA, WPA, SSA, SEC, FLSA)
	\end{itemize}
\end{frame}

\subsection{Another Renaissance}
\begin{frame}{Literary Renaissance}
	\begin{itemize}
		\item<+->W.E.B. Du Bois [1868--1963] (\emph{The Philadelphia Negro} [1899], \emph{The Souls of Black Folk} [1903])
		\item<+->Claude McKay [1889-1948] (``If We Must Die'' [1919])
		\item<+->e.e. cummings [1894--1962] (\emph{The Enormous Room} [1924], ``she being Brand'' [1926])
		\item<+->Langston Hughes [1902--1967] (``Jazz Band in a Parisian Cabaret'' [1925], ``Theme for English B'' [1951])
		\item<+->F. Scott Fitzgerald [1896--1940] (\emph{The Great Gatsby} [1925])
	\end{itemize}
\end{frame}

\subsection{The Rise of Jazz}
\begin{frame}{An American Music}
	\begin{itemize}
		\item<+->Ma Rainey [1886--1939]
		\item<+->Bessie Smith [1892--1937] (\emph{Florida-Bound Blues})
		\item<+->Duke Ellington [1899--1974] (\emph{It Don't Mean a Thing (If It Ain't Got That Swing)} [1932])
		\item<+->Louis Armstrong [1901--1971] (\emph{Hotter Than That} [1927])
		\item<+->George Gershwin [1898--1937] (\emph{Rhapsody in Blue} [1924], \emph{Porgy and Bess} [1935])
	\end{itemize}
\end{frame}

\subsection{Moving Arts}
\begin{frame}{Cinema}
	\begin{itemize}
		\item<+->1888. George Eastman develops a new camera (called the Kodak) and a small film: the first modern camera.
		\item<+->1890s. Thomas Edison uses that film (at the size of 35mm!) and creates a kinetoscope: the first video camera.
		\item<+->However, the films were only viewable by one person at a time and they were only 20 seconds long.
		\item<+->1895. August and Louis Lumi{\`e}re showed the first publicly viewable film on their own cin{\'e}matographe (20 minutes long).
		\item<+->1905. First nickelodeon (named for its price) theater.
		\item<+->By 1909, silent films are using intertitles to provide textual dialogue.
		\item<+->By 1915, silent films use a separate source of music as well.
		\item<+->1927. First talking movie was released (\emph{The Jazz Singer})
		\item<+->Studio system (Fox, MGM, Paramount, RKO, Warner, Universal, Columbia, United Artists).
		\item<+->Star system
		\item<+->Studios leverage stars to promote less-popular films.
	\end{itemize}
\end{frame}

\section{World at War}
\begin{frame}{The Great War}
% 	\slide{7}{Europe after the Great War}{map-1923-europe.pdf}
	
	\begin{itemize}
		\item<+->Germany punished brutally for WW1
		\item<+->1933. Rise of National Socialist party in Germany.
		\item<+->Fascist takeovers of Spain (Franco), Italy (Mussolini)
		\item<+->1927. Stalin takes control of Russia
		\item<+->Appeasement policy
		\item<+->Blitzkreig
		\item<+->The Holocaust
		\item<+->7 December 1941. Pearl Harbor
		\item<+->1 May 1945. V-E Day
		\item<+->6,9 August 1945. Atomic bombs on Hiroshima and Nagasaki
		\item<+->14 August 1945. V-J Day
	\end{itemize}
\end{frame}


