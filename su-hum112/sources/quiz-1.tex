\documentclass[12pt]{examdesign}
% \NoRearrange
\OneKey
\NumberOfVersions{3}
\class{HUM 112: World Cultures 2}
\examname{Quiz 1}
\begin{document}

% 6 questions = 24 points
\begin{shortanswer}[title={Short Answers}]
	The following questions can be answered with just one or two words. Each are worth 4 points.

	\begin{question}
		During the 14th century (1300s), the pope moved his residence from Rome to France in order to be closer to the French royal court. To which French town did he relocate?
		\examvspace{0.75 in}
		\begin{answer}
			Avignon
		\end{answer}
	\end{question}

	\begin{question}
		What is the name of the highly ornate style of art which developed later in the Renaissance?
		\examvspace{0.75 in}
		\begin{answer}
			Baroque
		\end{answer}
	\end{question}

	\begin{question}
		In which Italian city was the artistic development of the Renaissance centered?
		\examvspace{0.75 in}
		\begin{answer}
			Florence
		\end{answer}
	\end{question}

	\begin{question}
		Which council of the Roman Catholic Church convened as a response to the various Protestant Reformations?
		\examvspace{0.75 in}
		\begin{answer}
			Trent
		\end{answer}
	\end{question}

	\begin{question}
		What musical performance developed in Italy during the Renaissance which combined the use of text and music for the first time? (Hint: the name is plural because it consists of several smaller "works.")
		\examvspace{0.75 in}
		\begin{answer}
			Opera
		\end{answer}
	\end{question}

	\begin{question}
		What is the word which describes the cosmology as being centered around the \word{{earth} {sun}} rather than the \word{{sun} {earth}}?
		\examvspace{0.75 in}
		\begin{answer}
			\word{{Geocentrism} {Heliocentrism}}
		\end{answer}
	\end{question}
\end{shortanswer}

% 24 questions = 48 points
\begin{truefalse}[title={People}]
	In this section, you must identify in which field each person below is recognized. Please use the following: \textbf{A}rt, \textbf{M}usic, \textbf{P}olitics, \textbf{R}eligion,\textbf{L}iterature, \textbf{S}cience/Technology/Philosophy, \textbf{N}one. Yes, this means that some names below may not have lived during the Renaissance era. Each is worth 2 points.

	\begin{question}
		\answer{R} Martin Luther.
	\end{question}

	\begin{question}
		\answer{R} John Calvin.
	\end{question}

	\begin{question}
		\answer{A} Sandro Botticelli.
	\end{question}

	\begin{question}
		\answer{A} Leonardo da Vinci.
	\end{question}

	\begin{question}
		\answer{A} Donatello.
	\end{question}

	\begin{question}
		\answer{A} Raphael.
	\end{question}

	\begin{question}
		\answer{A} Caravaggio.
	\end{question}

	\begin{question}
		\answer{M} Gabrieli.
	\end{question}

	\begin{question}
		\answer{M} Monteverdi.
	\end{question}

	\begin{question}
		\answer{M} Vivaldi.
	\end{question}

	\begin{question}
		\answer{N} Madonna.
	\end{question}

	\begin{question}
		\answer{L} Geoffrey Chaucer.
	\end{question}

	\begin{question}
		\answer{L} William Shakespeare.
	\end{question}

	\begin{question}
		\answer{N} Wyclef Jean.
	\end{question}

	\begin{question}
		\answer{N} Isaac Newton.
	\end{question}

	\begin{question}
		\answer{R} Jan Hus.
	\end{question}

	\begin{question}
		\answer{P} Oliver Cromwell.
	\end{question}

	\begin{question}
		\answer{S} Copernicus.
	\end{question}

	\begin{question}
		\answer{S} Francis Bacon.
	\end{question}

	\begin{question}
		\answer{S} Galileo Galilee.
	\end{question}

	\begin{question}
		\answer{S} Johannes Kepler.
	\end{question}

	\begin{question}
		\answer{S} René Descartes.
	\end{question}

	\begin{question}
		\answer{N} Amadeus Mozart.
	\end{question}

	\begin{question}
		\answer{N} Thomas Aquinas.
	\end{question}
\end{truefalse}

\pagebreak
% 1 question = 8 points
\begin{shortanswer}[title={Short Essay}]
	The following question must be answered in one to two paragraphs. Use the back of this page if space is needed. It is worth 8 points.

	\begin{question}
		Briefly explain one major cultural event which occurred during the Renaissance era (we discussed at least four) and, in your own opinion, explain how it has changed our life today.
		\examvspace{5 in}
		\begin{answer}
			Freely.
		\end{answer}
	\end{question}
\end{shortanswer}

\pagebreak
\begin{shortanswer}[title={Bonus Section}, rearrange=no]
	The following questions are optional. You will receive an additional point for each one answered correctly. You cannot receive more than 80 points on this quiz, but any additional points can be applied to other grades which are below their maximum (e.g. participation/attendance, another quiz, an assignment).

	\begin{question}
		Which king of England broke with the Roman Catholic Church and began the Church of England?
		\examvspace{0.75 in}
		\begin{answer}
			Henry VIII
		\end{answer}
	\end{question}

	\begin{question}
		What is the name of the person (now a saint) who was imprisoned by the above king of England?
		\examvspace{0.75 in}
		\begin{answer}
			Thomas More
		\end{answer}
	\end{question}

	\begin{question}
		Who was king of the Holy Roman Empire during the beginning of Luther's Reformation?
		\examvspace{0.75 in}
		\begin{answer}
			Charles V
		\end{answer}
	\end{question}

	\begin{question}
		Which king united the crowns of Scotland and England? Give his full title.
		\examvspace{0.75 in}
		\begin{answer}
			James VI and I
		\end{answer}
	\end{question}

	\begin{question}
		Which king of Great Britain (Scotland and England) was overthrown and hung his own people?
		\examvspace{0.75 in}
		\begin{answer}
			Charles I
		\end{answer}
	\end{question}
\end{shortanswer}
\end{document}
