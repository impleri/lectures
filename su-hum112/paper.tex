\pagebreak
\section*{Assignment 2: Project Paper}
\emph{Due Week 8 and worth 200 points}

The Project Paper focuses on a suggested topic related to art, architecture, history, music, or literature. The project will reflect your views and interpretation of the topic. This project is designed to help you stretch your mind and your abilities to be the creative, innovative, and critical thinker you already are! Choose one (1) of the topics from the list of topic choices below. Read the topic carefully. Write a three to four (3--4) page paper (750--1,000 words) that responds to each of the items described in the topic.

\textbf{Note: Submit your topic choice to the instructor for approval before the end of Week 2.}

\subsection*{Topic Choices}
\begin{enumerate*}
	\item \textbf{Office Art Memo}. \emph{Memorandum}. Your boss, who knows you've been taking a humanities class since he pays for your tuition reimbursement, has tasked you with managing the art budget for your company, expecting you to choose various pieces of art for the new corporate offices.  (Note: Replicas of the works are acceptable since they are more cost-efficient and you are working on a budget.)  (1) Identify three examples of 19th century Impressionist painting or sculpture \textbf{and} three Post-Impressionist works. Explain how the six pieces of art fall into these two styles. (2) In a memo, describe the appearance of your six choices to your CEO so he or she will know what the art looks like and where it would be placed in the corporate offices. (3) Explain why each piece is considered to be historically significant. (4) Explain how each piece ``fits'' your company's overall (or desired) corporate image. Keep in mind that a piece of art is supposed to ``say'' something about the owner, so describe what would these pieces of art say about your company.

	\item \textbf{New Composition}. \emph{Speech}. Your uncle's birthday is in two months, and everyone knows that he loves almost all kinds of music. As a birthday gift for him, you want to have a special piece of music composed in his honor which will be played at a family birthday celebration. Write a speech that you will make to the composer's agent. (1) Narrow your choices down to three composers you've studied in this course. Choose one of the composers and explain why you want him to write the ``birthday present'' music. (2) Explain why the other two composers were ultimately not selected. (3) Specifically identify the musical elements in the composer's style that you would like to be included in the new music written for your uncle. Finally, (4) describe what sort of emotion is generated by listening to the works of your selected composer; in other words, what do you want your uncle to ``feel'' as he hears the music, and why is this composer so perfect for this composition?

	\item \textbf{Harlem Renaissance Poets}. \emph{Essay \& Poem}. Choose two poems by different authors from the Harlem Renaissance. Write an essay that (1) describes each author's role and importance within the Harlem Renaissance. (2) Identify the elements in each of their poems in which you see evidence of the ``double-consciousness'' being expressed by each author. (3) Describe the primary themes you see in the poetry written during this time period, referring to specific lines in each of the poems. Then, (4) write your own poem that expresses these identified themes of the Harlem Renaissance.

	\item \textbf{Women's Roles Then \& Now}. \emph{Script}. Script a conversation between two notable women from the 18th and/or 19th century on the roles women should play in society. Within the dialogue, include (1) biographical information for each woman, (2) the historical status for women in general during the time period in which each woman lived, (3) what opinions each of the women might have on the role the women should play in society during their lifetimes, and (4) what each of the women might think about women's current roles.

	\item Other topic choice recommended and \textbf{approved} by the professor and supported by the grading rubric.
\end{enumerate*}

Write a 3--4 page paper in which you:
\begin{itemize*}
	\item Support your ideas with specific, illustrative examples. If there are questions or points associated with your chosen topic, be sure to answer all of the listed questions and address all of the items in that topic. If your topic requires you to do several things related to the topic, be sure to do each of the things listed.
	\item While some of the topics tend to lend themselves toward particular writing genres, you are not restricted to the specific format suggested for the individual topic. For example, you may do an ``interview,'' a ``proposal,'' a ``letter,'' a ``short story,'' a ``blog,'' an ``essay,'' an ``article,'' or any other written genre for almost any of the topics. The project is intended to be fun as well as informative, so feel free to be creative with the delivery of your information.
	\item Use at least two (2) sources besides the textbook, which counts as one (1) source, for a minimum of three (3) sources.
\end{itemize*}

The specific course learning outcomes associated with this assignment are:
\begin{itemize*}
	\item Explain how key social, cultural, and artistic developments contribute to historical changes.
	\item Explain the importance of situating a society's cultural and artistic expressions within a historical context.
	\item Examine the influences of intellectual, religious, political, and socio-economic forces on social, cultural, and artistic expressions.
	\item Identify major historical developments in world cultures from the Renaissance to the contemporary period.
	\item Use technology and information resources to research issues in the study of world cultures.
	\item Write clearly and concisely about world cultures using proper writing mechanics.
\end{itemize*}

\rubricForty{
	Follow topic instructions within the individual topic selected.
	& Did not follow instructions; omitted information and/or included irrelevant information.
	& Partially followed instructions; omitted some key information.
	& Sufficiently followed instructions.
	& Fully followed instructions. \\ \hline

	Address each of the four (4) points in the chosen topic.
	& Did not address each of the four points; omitted information and/or included irrelevant information.
	& Partially addressed each of the four points; omitted some key information.
	& Sufficiently addressed each of the four points.
	& Fully addressed each of the four points. \\ \hline

	Provide sufficient information, examples, and details to support the general claim or main idea.
	& Did not provided adequate information; omitted information and/or included irrelevant information.
	& Partially provided sufficient information; omitted some key information.
	& Sufficiently provided adequate information.
	& Fully provided sufficient information.
}
