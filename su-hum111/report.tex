\pagebreak
\section*{Assignment 3: Cultural Event Report}
\emph{Due Week 10 and worth 100 points}

As a way of experiencing the Humanities beyond your classroom, computer, and textbook, you are asked to attend a ``cultural event'' and report on your experience. This could be visiting a museum or gallery exhibition; attending a theater, dance, or musical performance; or something similar. You should attend this event before the end of Week 10. Write a two to three (2--3) page report (500--750 words) that describes your experience.

\textbf{Note: Submit your cultural event choice to the instructor for approval before the end of Week 5.}

\subsection*{Event Choices}
\begin{enumerate*}
	\item Visiting a Museum
		\begin{itemize*}
			\item It makes sense to approach a museum the way a seasoned traveler approaches visiting a city for the first time. Find out what there is available to see. In the museum, find out what sort of exhibitions are currently housed in the museum and start with the exhibits that interest you.
			\item If there is a travelling exhibition, it's always a good idea to see it while you have the chance. Then, if you have time, you can look at other things in the museum.
			\item Make notes as you go through the museum and accept any handouts or pamphlets that the museum staff gives you. While you should not quote anything from the printed material when you do your report, the handouts may help to refresh your memory later.
			\item The quality of your experience is not measured by the amount of time you spend in the galleries or the number of works of art that you actually see. The most rewarding experiences can come from finding one or two (1 or 2) pieces of art or exhibits which intrigue you and then considering those works in leisurely contemplation. Most museums even have benches where you can sit and study a particular piece.
			\item If you are having a difficult time deciding which pieces to write about, ask yourself these questions: (1) If the museum you are visiting suddenly caught fire, which two (2) pieces of art or exhibits would you most want to see saved from the fire? (2) Why would you choose those two (2) particular pieces?
		\end{itemize*}
	\item Attending a Performance
		\begin{itemize*}
			\item Check your local colleges to see if there are any free or low-cost performances or student recitals. Student performances are generally of almost the same quality as professional performances, but typically cost much less.
			\item Unlike visiting a museum, where you can wear almost anything, people attending performances are often expected to ``dress up'' a bit.
			\item Take a pen or pencil with you and accept the program you are offered by the usher; you will probably want to make notes on it during or after the performance.
			\item Turn off your cell phone before entering the auditorium. Do not use your phone to record the music or to take pictures or videos. To play it safe, turn the phone off.
			\item Most long musical performances have at least one (1) intermission. If the lights start blinking, it is a sign that the performance is about to begin.
			\item Look for very specific things (such as a particular piece of music or the way certain instruments sounded at a specific time) which tend to stand out as either enjoyable or not enjoyable. Be sure to make notes of the things which you find enjoyable as well as the things which are not enjoyable.
		\end{itemize*}
\end{enumerate*}

The specific course learning outcomes associated with this assignment are:
\begin{itemize*}
	\item Explain the importance of situating a society's cultural and artistic expressions within a historical context.
	\item Examine the influences of intellectual, religious, political, and socio-economic forces on social, cultural, and artistic expressions.
	\item Use technology and information resources to research issues in the study of world cultures.
	\item Write clearly and concisely about world cultures using proper writing mechanics.
\end{itemize*}

\rubricTwenty{
	Clearly identify the event location, date attended, the attendees, and your initial reaction upon arriving at the event.
	& Did not identify the event details; omitted information and/or included irrelevant information.
	& Partially identified the event; omitted some key information.
	& Sufficiently identified the event.
	& Fully identified the event. \\ \hline

	Provide specific information and a description of at least two (2) pieces (e.g. art, exhibits, music, etc.).
	& Did not provide specific information; omitted information and/or included irrelevant information.
	& Partially provided specific information; omitted some key information.
	& Sufficiently provided specific information.
	& Fully provided specific information. \\ \hline

	Provide a summary of the event and describe your overall reaction after attending the event.
	& Did not provide a summary nor a reaction; omitted information and/or included irrelevant information.
	& Partially provided a summary and/or reaction; omitted some key information.
	& Sufficiently provided a summary and your overall reaction.
	& Fully provided a summary and your overall reaction.
}

