\pagebreak
\section*{Assignment 2: Project Paper}
\emph{Due Week 8 and worth 200 points}

The Project Paper focuses on a suggested topic related to art, architecture, history, music, or literature. The project will reflect your views and interpretation of the topic. This project is designed to help you stretch your mind and your abilities to be the creative, innovative, and critical thinker you already are! Choose one (1) of the topics from the list of topic choices below. Read the topic carefully. Write a three to four (3--4) page paper (750--1,000 words) that responds to each of the items described in the topic.

\textbf{Note: Submit your topic choice to the instructor for approval before the end of Week 2.}

\subsection*{Topic Choices}
\begin{enumerate*}
	\item \textbf{Self-Portraits}. \emph{Journal}. The Renaissance artists Titian, Rembrandt, and Durer have each painted self-portraits. Imagine yourself as one of these artists (or another artist of your choice that has painted a self-portrait) and write a journal expressing your thoughts on ``your'' art (in other words, the journal entries the artist would probably write.) (1) Explain ``your'' primary reason for painting a self-portrait. (2) Describe ``your'' artistic choices in composition: use of color, space, etc. (3) Explain what the portrait represents about ``you'' (as the artist.) (4) Explain ``your'' choices of subject with regard to at least three other specific pieces ``you'' have painted.

	\item \textbf{Ladies \& Gentlemen}. \emph{Survey \& Report}. Some may believe that being ``gentleman'' or a ``lady'' in today's society is an outdated notion, but others may disagree. (1) Briefly summarize the main characteristics of a well-rounded person, ``l'uomo universal,'' referring to specific sections within the excerpt from \emph{The Courtier} which identifies these characteristics. (2) Create a ``survey'' based on the identified characteristics and ``poll'' at least ten people to find out whether or not the characteristics are relevant for a ``gentleman'' or ``lady'' of today. (3) Tabulate and discuss the responses in terms of gender, age, vocation, etc. of your survey participants, making note of any interesting or surprising results which show up in your poll answers. (4) Finally, explain whether or not you agree with the characteristics expressed by Castiglione in \emph{The Courtier}.

	\item \textbf{Bayeux Tapestry Experience}. \emph{Letter}. Imagine yourself as one of the figures in the battle depicted in the Bayeux Tapestry; in a letter home, you describe your experience to your family. (1) Write a first-person account of this historical event from the perspective of one of the figures in tapestry. (2) Use your senses to describe your impression of the event. (3) Describe specific elements of the scene such as uniforms, weaponry, fighting styles, etc. (4) Explain why you believe your side was justified in participating in the battle and how you would like the battle to be remembered.

	\item \textbf{Ancient Chinese Contributions}. \emph{Essay}. To win a trip to China, you enter a contest to determine the four most useful contributions or inventions created by the ancient Chinese. (1) Identify eight to ten of these useful inventions or contributions. (2) Nominate four that you believe are the most ingenious or innovative. (3) Explain why you believe these four inventions or contributions are the most useful inventions or contributions from the ancient Chinese. (4) Identify one invention or contribution that you cannot live without and explain why.

	\item Other topic choice recommended and approved by the professor and supported by the grading rubric.
\end{enumerate*}

Write a 3--4 page paper in which you:
\begin{itemize*}
	\item Support your ideas with specific, illustrative examples. If there are questions or points associated with your chosen topic, be sure to answer all of the listed questions and address all of the items in that topic. If your topic requires you to do several things related to the topic, be sure to do each of the things listed.
	\item While some of the topics tend to lend themselves toward particular writing genres, you are not restricted to the specific format suggested for the individual topic. For example, you may do an ``interview,'' a ``proposal,'' a ``letter,'' a ``short story,'' a ``blog,'' an ``essay,'' an ``article,'' or any other written genre for almost any of the topics. The project is intended to be fun as well as informative, so feel free to be creative with the delivery of your information.
	\item Use at least two (2) sources besides the textbook, which counts as one (1) source, for a minimum of three (3) sources.
\end{itemize*}

The specific course learning outcomes associated with this assignment are:
\begin{itemize*}
	\item Explain how key social, cultural, and artistic developments contribute to historical changes.
	\item Explain the importance of situating a society's cultural and artistic expressions within a historical context.
	\item Examine the influences of intellectual, religious, political, and socio-economic forces on social, cultural, and artistic expressions.
	\item Identify major historical developments in world cultures during the eras of antiquity to the Renaissance.
	\item Use technology and information resources to research issues in the study of world cultures.
	\item Write clearly and concisely about world cultures using proper writing mechanics.
\end{itemize*}

\rubricForty{
	Follow topic instructions within the individual topic selected.
	& Did not follow instructions; omitted information and/or included irrelevant information.
	& Partially followed instructions; omitted some key information.
	& Sufficiently followed instructions.
	& Fully followed instructions. \\ \hline

	Address each of the four (4) points in the chosen topic.
	& Did not address each of the four points; omitted information and/or included irrelevant information.
	& Partially addressed each of the four points; omitted some key information.
	& Sufficiently addressed each of the four points.
	& Fully addressed each of the four points. \\ \hline

	Provide sufficient information, examples, and details to support the general claim or main idea.
	& Did not provided adequate information; omitted information and/or included irrelevant information.
	& Partially provided sufficient information; omitted some key information.
	& Sufficiently provided adequate information.
	& Fully provided sufficient information.
}
