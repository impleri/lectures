% Marx and Political Theology

\begin{enumerate}
 	\item The Death of God
	\begin{itemize}
		\item Marx's philosophy follows from his understanding of Hegel
		\item While Marx does very little to explicitly lay out a vision of religion, he does imply both how he saw its role in the past and how he hoped it'd become in the future
		\item The death of God is generally ascribed to Nietzsche by modern readers, but it was in fact Hegel which first proposed that God had died. It is this notion that 'haunts' Marx's project.
		\item For Hegel, the individualism that had arisen in Christianity has alienated people from the communal view because Xtnity has 'objectified the divine, set it over against this world and introduced a religion of otherworldliness and transcendence' (MacIntyre, Marxism and Christianity, 11). Cf Nietzsche, The Antichrist.
		\item For Hegel, the figure of Jesus was against this alienation and, instead, preaching an interiority to religion which reconciles the estrangement of the human with humanity.
		\item This resolution takes the form of freedom, such that the goal is free humans in a free society.
		\item Marx takes his understanding of history from this conception, and it is philosophy that rescues the truth of religion (i.e. this freedom) from the relationship of control and subjugation that exists.
		\item However, Hegel is highly optimistic about this ideal. Marx approaches Hegel with suspicion because reality and history have not upheld Hegel's idealism.
		\item When one reads Marx, one should keep in mind that Marx attempts to be very practical, applying the Hegelian ideal to reality such that each conform to the other.
		\item What I propose is to follow Marx in layers, from one core aspect of his thought outward. I will start with his 'mature' theory of commodities (from Capital 1) and move into his concept of fetishism (what I assigned for reading). From there, I will move outward and point out two closely related critiques: of capital and of the political. Thirdly, I shall move further away and talk of Marx's understanding and critique of religion. Fourthly, I will point out a few 'modern' takes on Marxism, touching on a few recent articles (including the second piece of reading I assigned). Finally, I will hone in on what a Marxist political theology might look like today.
	\end{itemize}
 	\item Commodity fetishism (Capital)
	\begin{itemize}
		\item Use-value = usefulness (physical body of the commodity), exchange-value = proportion in which one commodity can be exchanged for another
		\item All value can be reduced to the production of labour such that labour is the basis of exchange. 'As exchange-values, all commodities are merely definite quantities of congealed labour-time'.
		\item Therefore, the substance of value is labour. The measure of the magnitude of value is labour-time.
		\item The value of a commodity can only be expressed as a quantity of another commodity.
		\item A commodity may become the universal equivalent, that by which all other commodities are valued. However, this commodity must be excluded from the market, lest it serves as its own equivalent. That is, the value of a universal equivalent is tautological and does not actually express an exchange-value.
		\item The universal equivalent serves as money, turning the value of all other commodities into their commodity-equivalent.
		\item The natural form of labour is social and based on exchange-value.
		\item Wealth is a use-value, a property, of man. It has no exchange-value.
	\end{itemize}
 	\item Marx's critique of capital (Manifesto)
	\begin{itemize}
		\item As labour becomes externalised (through the fetishism of commodities), the human itself becomes a commodity. This leads to self-alienation because one begins to see oneself as a bare minimum of life in order to continue being valuable (e.g. 'live to work')
		\item This estrangement makes one reflective. In this sense, estrangement is a necessary part of the human life. However, difficulties arise when one breaks with the social community. This leads to some humans gaining power over others by means of exchange. This is the elementary form of class that has occurred throughout history.
		\item Struggle is between these two classes: capitalist/bourgeois and worker/proletariat. Because of the capitalist system, the workers always lose (if the capitalist loses money, the workers lose money...if the capitalist gains money, he replaces the workers with machines. As a result, the capitalist gains and loses only capital/money. However, the worker can lose her chance of a livelihood and, indeed, of properly human life.
		\item Everything can be subsumed into capitalism, even this critique. Capital = accumulated wealth = accumulated labour = labour.
		\item Capitalism denies private property to most, 'crude communism' denies it to all. Marx wants to overthrow both so that instead of a society which denies most or all the possibility of being human, socialism represents giving all the possibility of humanity. This is where the society satisfies the demands of the people not the economic system. Humanity must become social.
		\item This can only occur if the workers unite against the capitalists and revolt, abolishing the capitalist system of economics. In Marx's eyes, workers have nothing to lose but their chains (of slavery to capital).
	\end{itemize}
 	\item Marx's critique of religion
	\begin{itemize}
		\item The religion of crude communism is atheism. The religion of socialism is apathy. Religion, for Marx, is connected to political power as its own capitalist system of exchange-value. In a socialist society, because the political is transformed, there will be nothing to fuel religion. Therefore, Marx believes religion will wither away on its own.
	\end{itemize}
 	\item Marxism and Christianity
	 \item A Lacanian twist
 	\item The Last Prophet
\end{enumerate}
