\documentclass[11pt]{article}
\usepackage{fullpage}
\usepackage[english]{babel}
\usepackage[babel=true]{csquotes}
\pagestyle{plain}
\clubpenalty=10000
\widowpenalty=10000

\begin{document}
\begin{enumerate}
	\item Multiple influences (19th century)
		\begin{itemize}
			\item Insularity of religious academ\'e (e.g. Princeton's \enquote{Old Guard} redefining orthodoxy in terms of Common Sense Realism)
			\item The overall displacement of religion and theology from the academic and social scene throughout the Enlightenment and Industrial Revolution (e.g. higher criticism)
			\item Darwin's theory of evolution became the straw that broke the camel's back
			\item Influx of Catholic immigrants (esp. Irish and Italian)
		\end{itemize}

	\item With the fear of extinction before them, some of the popular American Protestant groups began to join together at the end of the nineteenth century as a reaction to these changes (cf. Wahhabis in Saudi Arabia just a few decades before). These Christians began meeting annually to discuss what would be the central tenets for Christians according to them. During the 1910s, a series of essays which were eventually collected into a single book --- \emph{The Fundamentals} --- became the defining feature of their movement; and by 1922, the term \textbf{fundamentalism} was in common usage to describe this group. To these fundamentalists, they were defending Christian belief in America from the erosion of faith caused by newer \enquote{innovations} in thought --- from higher criticism (now considered a basic stepping stone in biblical studies) to Catholicism (immigration), socialism, atheism, and the newer religious movements (e.g. Mormons, Jehovah's Witnesses). The movement's most iconic moment, however, was the infamous Scopes \enquote{monkey} trial in Tennessee. Even though the fundamentalists won the day, their popularity quickly waned and they disappeared from the social view.

	\item Political (and cultural) effects of fundamentalism
		\begin{itemize}
			\item The same groups committed to the fundamentals were also committed to temperance and prohibition (which was ratified in 1920).
			\item Creationism as a \emph{scientific} concept.
			\item A stronger divide between fundamentalists and \enquote{mainline} (e.g. Anglican, Catholic) Christian groups over new \enquote{lines in the sand} such as inerrancy and millennialism.
			\item A distrust of contemporary science (see creationism above).
			\item At the same time as the rise of (historic) fundamentalism, the American political scene was seeing a resurgence in conservative values (over a 30+ year period) dominated by Republican presidents who shifted (thanks to a split by TR for the Bull Moose Party) towards the political right.
		\end{itemize}

	\item During the 1940s and 50s, a divide in the fundamentalist camp emerged between those who wanted to remove the general militancy from their beliefs. These less militant Protestants began to identify themselves as \textbf{evangelical} to emphasize the less-militant, more personal aspects of Christian fundamentalism. An important figure at this time was a young Billy Graham. Slowly, evangelicals began to absorb liberal theology (which focuses on the individual's experiences). From the evangelical movement emerged a new way of talking about Christian belief: \enquote{born again} faith which gained currency in the late 70s and throughout the 80s.

	\item By the 1960s (with the Civil Rights movement), evangelicals (and fundamentalists) who, along with many conservative Republicans, opposed integration began to align themselves politically on one side of the spectrum to the exclusion of others. By the late 1970s, the Moral Majority headed by Jerry Falwell and Pat Robertson as well as the Family Research Council (James Dobson) and Concerned Women for America (Beverly LaHaye) formed the nucleus of the Christian Right.

	\item Today, the political and social effects of this one strand of a single religion are manifold.
		\begin{itemize}
			\item I've already mentioned the rise of creationism as a scientific concept; the battle depicted between Christians defending the very truth of God and the godless atheist scientists hell-bent on advocating their own \enquote{religion} (and the alternate image of educated, impartial scientists rooting out ignorant, archaic \enquote{belief} that is a relic of a religious past) is a common one now.
			\item We can also see an effect in political figures like Christine O'Donnell and the advocacy of abstinence until marriage. Other thoughts around sexuality such as adultery (which has cost Herman Cain but not Newt Gingrich) and homosexuality (remember Ted Haggard?) have also become prominently associated with \enquote{conservative} values both politically and theologically.
			\item Unwavering support for the state of Israel (due to millennial beliefs).
			\item Staunch support of modern capitalism contra Marxist/socialist tendencies (see, e.g., Conservapedia's Conservative Bible Project).
		\end{itemize}

	\item Question: Can you think of figures who claim an evangelical faith but do not fit the conservative mold? Jimmy Carter, Bill Clinton.
	\item Question: Can you imagine a conservative evangelical's response to the current string of OccupyX chapters? Do you think religious language would help or hurt that movement?
\end{enumerate}

\end{document}